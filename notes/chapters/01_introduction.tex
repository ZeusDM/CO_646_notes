\chapter{Matroids}

\section{Definition}

\begin{definition}[Matroid]
    A (finite) \vocab{matroid} is a pair \( (E, \mathcal{I}) \) where
    \(E\) is a finite set, called the \vocab{ground set}, and
    \(\mathcal{I} \subseteq 2^E\) is a collection of subsets of \(E\) such that
    \begin{enumerate}[label = \textup{(I\arabic*)}]
        \item \(\mathcal{I} \neq \emptyset\), \label{item:matroid:nonempty}
        \item for all \(I \in \mathcal{I}\) and \(J \subseteq I\), we have \(J \in \mathcal{I}\), \label{item:matroid:hereditary}
        \item for all \(I, J \in \mathcal{I}\) with \(|I| < |J|\), there exists \(e \in J \setminus I\) such that \(I \cup \{e\} \in \mathcal{I}\). \label{item:matroid:exchange}
    \end{enumerate}
\end{definition}

\begin{definition}[Basis]
    A \vocab{basis} for a set \(X \subseteq E\) in a matroid \(M = (E, \mathcal{I})\) is a maximal independent subset of \(X\).
    A basis for \(M = (E, \mathcal{I})\) is a basis for \(E\) in \(M\).
\end{definition}

\begin{note}
    Every independent set \(I \subseteq X\) is contained in a basis for \(X\).
\end{note}
One can consider infinite matroids by changing \ref{item:matroid:exchange} and making this note an axiom.

\begin{proposition}
    For an \(X \subseteq E\) in a matroid \(M = (E, \mathcal{I})\), all bases for \(X\) have the same size.
\end{proposition}

\begin{proof}
    Follows from \ref{item:matroid:exchange}.
\end{proof}

\begin{definition}[Rank]
    The \vocab{rank} of a set \(X \subseteq E\) in \(M\) is the size of a basis for \(X\).
    The \vocab{rank} of a matroid \(M = (E, \mathcal{I})\) is the rank of \(E\) in \(M\), i.e., the size of a basis for \(M\).
\end{definition}

\begin{proposition}
    If \((E, \mathcal{I})\) satisfies \ref{item:matroid:nonempty}, \ref{item:matroid:hereditary}, and
    \begin{enumerate}[label = \textup{(I\arabic*)}, start = 4]
        \item for all \(X \subseteq E\), all maximal sets in \(\mathcal{I} \cap 2^X\) have the same size, \label{item:matroid:maximal}
    \end{enumerate}
    then \((E, \mathcal{I})\) is a matroid.
\end{proposition}

\begin{proof}
    We need to show that \ref{item:matroid:exchange} holds.
    Let \(I, J \in \mathcal{I}\) with \(|I| < |J|\).
    Let \(X = I \cup J\).
    By \ref{item:matroid:maximal}, \(I\) is not a maximal set in \(\mathcal{I} \cap 2^X\), therefore there exists \(e \in X \setminus I = J \setminus I\) such that \(I \cup \{e\} \in \mathcal{I}\).
    Therefore, \ref{item:matroid:exchange} holds.
\end{proof}

\section{Representable Matroids}

Let \(\mathbb{F}\) be a field and \(A \in \mathbb{F}^{R \times E}\) be a matrix.
Let
\begin{equation}
    \mathcal{I} = \{I \subseteq E : \text{the columns of } A[I] \text{ are linearly independent}\}.
\end{equation}
Then \((E, \mathcal{I})\) is a matroid.

\begin{proof}
    \ref{item:matroid:nonempty} holds because the empty set is linearly independent.
    \ref{item:matroid:hereditary} holds because any subset of a linearly independent set is linearly independent.
    \ref{item:matroid:maximal} holds since, for each \(X \subseteq E\), the maximal sets in \(I \cap 2^X\) are the bases for the column space of \(A[X]\) contained in \(X\), which all have size equal to the rank of \(A[X]\).
\end{proof}

We write \(M(A)\) for the matroid \((E, \mathcal{I})\) obtained above for a matrix \(A\).
A matroid of the form \(M(A)\) for some \(A \in \mathbb{F}^{R \times E}\) is called \vocab{\(\mathbb{F}\)-representable} (a.k.a. linear over \(\mathbb{F}\), for the algebra people).

A \vocab{matrix} \(A\) in \(\mathbb{F}^{R \times E}\) is a function \(A \colon R \times E \to \mathbb{F}\).
Note that rows and columns have no ordering.

\begin{note}
    If \(M\) is \(\mathbb{F}\)-representable, and \(\mathbb{F} \subseteq \mathbb{F}'\), then \(M\) is also \(\mathbb{F}'\)-representable.
\end{note}

\begin{question}
    Given \(\mathbb{F}\), which matroids are \(\mathbb{F}\)-representable?
\end{question}

\begin{question}
    Can different matrices represent the same matroid?
\end{question}

\begin{proof}[Answer]
    Yes! Elementary row operations, removing or appending redundant rows, and non-zero column scalings preserve the matroid. Also field automorphisms.
    Still, these operations are not enough. For example, the matrices
    \begin{equation}
        \begin{bmatrix}
            1 & 1 & 1 & 1 \\
            0 & 1 & 2 & 3
        \end{bmatrix}
        \quad \text{and} \quad
        \begin{bmatrix}
            1 & 1 & 1 & 1 \\
            7 & 41 & 29 & 87
        \end{bmatrix}
    \end{equation}
\end{proof}

\begin{question}
    Is every matroid \(\reals\)-representable?
\end{question}

\begin{proof}[Answer]
    No.
    The Fano matroid \(F_7\) is formed by taking the Fano plane's points as the ground set,
    and the three-element noncollinear subsets as bases.
    The Fano matroid \(F_7\) is not \(\reals\)-representable.
    More generally, if the Fano matroid \(F_7\) is \(\mathbb{F}\)-representable,
    then it can be shown that \(\mathbb{F}\) has characteristic \(2\).
\end{proof}

\begin{figure}[htbp]
    \centering
    \begin{tikzpicture}[ mydot/.style={ draw, circle, fill=black, inner sep=1.5pt} ]
        \draw (0,0) coordinate (A) -- (2,0) coordinate (B) -- ($ (A)!.5!(B) ! {sin(60)*2} ! 90:(B) $) coordinate (C) -- cycle;
        \coordinate (O) at (barycentric cs:A=1,B=1,C=1);
        \draw (C) -- ($ (A)!.5!(B) $) coordinate (LC); 
        \draw (A) -- ($ (B)!.5!(C) $) coordinate (LA); 
        \draw (B) -- ($ (C)!.5!(A) $) coordinate (LB); 
        \draw (O) circle [radius=0.56cm];
        \foreach \Nodo in {A,B,C,O,LC,LA,LB} \node[mydot] at (\Nodo) {};    
    \end{tikzpicture} 
    \caption{The Fano plane.}
    \label{fig:fano}
\end{figure}

\section{Graphic Matroids}

Let \(G = (V, E, \sim)\) be a (multi)graph.
Let \(\mathcal{I} = \{ I \subseteq E : G[I] \text{ is acyclic} \}\).
Then \((E, \mathcal{I})\) is a matroid.

\begin{proof}
    \ref{item:matroid:nonempty} holds because the empty set is acyclic.
    \ref{item:matroid:hereditary} holds because any subset of an acyclic graph is acyclic.
    \ref{item:matroid:maximal} holds since each spanning tree of each component of \(G[X]\) has the same size for all \(X \subseteq E\).
\end{proof}

We write \(M(G)\) for the matroid \((E, \mathcal{I})\) obtained above for a graph \(G\),
called the \vocab{cycle matroid} of \(G\).
If \(M\) is the cycle matroid of a graph, then we say that \(M\) is a \vocab{graphic matroid}.

\begin{question}
    Is every representable matroid a graphic matroid?
\end{question}

\begin{proof}[Answer]
    No.
    Consider the representable matroid
    \begin{equation}
        M\left(
            \begin{bmatrix} 
                1 & 1 & 1 & 1 & \cdots & 1 \\
                1 & 2 & 3 & 4 & \cdots & n 
            \end{bmatrix}
        \right)
        = U_{2, n}.
    \end{equation}
    If it was a graphic matroid,
    then the graph would have no loops,
    no pair of parallel edges,
    and any three distinct edges contains a cycle.
    This is impossible for \(n \geq 4\).
\end{proof}

\begin{proposition}
    Let \(\mathbb{F}\) be a field.
    Every graphic matroid is \(\mathbb{F}\)-representable.
\end{proposition}

\begin{proof}
    Let \(G = (V, E, \sim)\) be a multigraph.
    Arbitrarily orient the non-loop edges of \(G\) so that each non-loop edge \(e\) has a tail and a head.
    Let \(A \in \mathbb{F}^{V \times E}\) be the incidence matrix of \(G\).
    \begin{equation}
        A_{v, e} = \begin{cases}
            1 & \text{if } v \text{ is the tail of } e \\
            -1 & \text{if } v \text{ is the head of } e \\
            0 & \text{otherwise.}
        \end{cases}
    \end{equation}
    The matrix \(A\) is a \vocab{signed incidence matrix} of \(G\).
    We will prove that \(M(A) = M(G)\).

    First, we claim that, if \(G[I]\) is acyclic, then the columns of \(A[I]\) are linearly independent. (That is, \(\mathcal{I}(M(G)) \subseteq \mathcal{I}(M(A))\).)
    \textit{Proof of the claim.}
    Let \(I\) be a minimal counterexample, so there exists coefficients \(c_e \in \mathbb{F}\) such that \(\sum_{e \in I} c_e A_{e} = 0\).
    By minimality, all \(\lambda_e\) are nonzero.
    But since \(G[I]\) is a nonempty forest, there exists a vertex \(w\) incident to exactly one edge \(f\) in \(I\) (a.k.a., a leaf).
    Therefore, \(0 = 0_w = (\sum_{e \in I} c_e A_{e})_{w} = (c_fA_{f})_w = \pm c_f \neq 0\), a contradiction.

    Now, we claim that, if \(G[X]\) contains a cycle \(C\), then \(A[X]\) has linearly dependent columns. (That is, \(2^E - \mathcal{I}(M(G)) \subseteq 2^E - \mathcal{I}(M(A))\).)
    \textit{Proof of the claim.}
    The \(V(C) \times E(C)\) submatrix of \(A\) is a square matrix whose rows sum to zero.
    Therefore, its rows are linearly dependent, and consequently, its columns are linearly dependent.
    Therefore, \(A[E(C)]\) has linearly dependent columns.

    Hence, we have \(\mathcal{I}(M(G)) = \mathcal{I}(M(A))\), which implies that \(M(G)\) is \(\mathbb{F}\)-representable.
\end{proof}

\begin{note}
    If \(\mathbb{F}\) has characteristic \(2\), then \(1 = -1\), so orientations are not needed.
\end{note}

\begin{note}
    The rows of \(A(G)\) always sum to zero, so they are linearly dependent.
\end{note}

\begin{note}
    If \(\operatorname{char}(\mathbb{F}) \neq 2\), we really do need signs, otherwise we get a (usually different) matroid that depends on the field.
\end{note}

\section{Terminology}

\begin{itemize}
    \item A matroid that is representable over \emph{every} field is \vocab{regular}.
    \item A \vocab{circuit} of a matroid is a minimal dependent set.
    Note that a circuit of \(M(G)\) is precisely the edge set of a cycle in \(G\).
    \item Circuits of sizes \(1\), \(2\) and \(3\) are called \vocab{loops}, \vocab{parallel pairs} and \vocab{triangles}, respectively.
\end{itemize}

\begin{note}
    Being parallel (equal or a parallel pair) is an equivalence relation.
\end{note}

\section{Rank Function and Properties}

The rank \(r_M(X)\) is the size of each basis for \(X\) in \(M\).
We have
\begin{enumerate}[label = \textup{(R\arabic*)}]
    \item \(0 \leq r_M(X) \leq |X|\) for all \(X \subseteq E\), \label{item:rank:bounded}
    \item \(r_M(X) = r_M(Y)\) if \(X\) and \(Y\) are independent sets, \label{item:rank:motonocity}
    \item \(r_M(X) + r_M(Y) \leq r_M(X \cup Y) + r_M(X \cap Y)\) for all \(X, Y \subseteq E\), \label{item:rank:submodularity}
    
    \begin{proof}[Proof of \ref{item:rank:submodularity}]
        Let \(I\) be a basis for \(X \cap Y\), and let \(J\) be a basis for \(X \cup Y\) containing \(I\).
        Note that \(J \cap X\) is an independent set in \(X\) and \(J \cap Y\) is an independent set in \(Y\),
        therefore \(r_M(X) \geq |J \cap X|\) and \(r_M(Y) \geq |J \cap Y|\).
        Therefore, we have
        \begin{align*}
            r_M(X) + r_M(Y) &\geq |J \cap X| + |J \cap Y| \\
            &= |J \cap (X \cup Y)| + |J \cap (X \cap Y)| \\
            &\geq |J| + |I| \\
            &= r_M(X \cup Y) + r_M(X \cap Y). \qedhere
        \end{align*}
    \end{proof}
\end{enumerate}

\begin{proposition} \label{prop:rank:newaxioms}
    If \(r \colon 2^E \to \nonnegatives\) satisfies
    \begin{enumerate}[label = \textup{(r\arabic*)}]
        \item \(r(\varnothing) = 0\), \label{item:rank:empty}
        \item \(r(X + e) \leq r(X) + 1\) for all \(X \subseteq E\) and \(e \in E - X\), \label{item:rank:step}
        \item \(r(X + e) + r(X + f) \geq  r(X + e + f) + r(X)\) for all \(X \subseteq E\) and \(e, f \in E - X\) (possibly \(e = f\)), \label{item:rank:submodularity2}
    \end{enumerate}
    then \((E, \{I \subseteq E : r(I) = |I|\})\) is a matroid whose rank function is \(r\).
\end{proposition}

This will be proved later.

\begin{lemma}[``Diminishing returns''] \label{lemma:diminishing}
    Assume that \(r \colon 2^E \to \nonnegatives\) satisfies \ref{item:rank:empty}, \ref{item:rank:step}, and \ref{item:rank:submodularity2}.
    Then,
    \begin{equation}
        r(X + e) - r(X) \geq r(Y + e) - r(Y)
    \end{equation}
    for all \(X \subseteq Y \subseteq E\) and \(e \in E\).
\end{lemma}

\begin{proof}
    The proof is by induction on \(|Y - X|\).
    If \(|Y - X| = 0\), then \(Y = X\) and the result holds.
    Otherwise, there exists \(f \in Y - X\).
    If \(e \in X \subset Y\), then 
    \begin{equation}
        r(X + e) - r(X) = 0 \geq 0 = r(Y + e) - r(Y).
    \end{equation}
    If \(e \in Y - X\), then
    \begin{align*}
        r(X + e) - r(X)
        &\geq r((X+f)+e) - r(X + f) & \text{(by \ref{item:rank:submodularity2})} \\
        &\geq r(Y + e) - r(Y) & \text{(by induction hypothesis)}.
    \end{align*}
\end{proof}

\begin{proof}[Proof of \ref{prop:rank:newaxioms}]
    \ref{item:matroid:nonempty} is trivial since \(\varnothing \in \mathcal{I}\).
    \ref{item:rank:step} and induction gives 
    \begin{equation}
        r(X \cup A) \leq r(X) + |A|.
    \end{equation}
    If \(J \in \mathcal{I}\) and \(I \subseteq J\), then
    \begin{equation}
        |J| = r(J) \leq r(I) + |J - I|,
    \end{equation}
    and consequently \(r(I) \geq |I|\).
    Also, \(r(I) = r(\varnothing \cup I) \leq r(\varnothing) + |I| = |I|\).
    Therefore, \(|I| = r(I)\) and consequently \(I \in \mathcal{I}\).
    Thus, \ref{item:matroid:hereditary} holds.

    Now we claim that, for \(X \subset E\), each maximal set in \(\mathcal{I} \cap 2^X\) has \(|I| = r(X)\).
    \textit{Proof of the claim.}
    Let \(Y\) be maximal such that \(I \subset Y \subset X\) and \(r(Y) = |I|\).
    If \(Y = X\), we are done.
    Otherwise, let \(f \in  X - Y\)
    By maximality of \(Y\), \(r(Y + f) > r(Y)\), and consequently
    \begin{equation}
        0 < r(Y + f) - r(Y) \leq r(I + f) - r(I) \leq 0,
    \end{equation}
    so \ref{item:matroid:maximal} holds.

    Therefore, \((E, \mathcal{I})\) is a matroid and the rank function is \(r\) by the previous claim.
\end{proof}

Since the rank function of any matroid satisfies \ref{item:rank:empty}, \ref{item:rank:step}, and \ref{item:rank:submodularity2} (it's not hard to show), this means that \ref{item:rank:empty}, \ref{item:rank:step}, and \ref{item:rank:submodularity2} give an alternative definition of matroids.

\begin{note}
    \ref{item:rank:bounded}, \ref{item:rank:motonocity}, and \ref{item:rank:submodularity} imply \ref{item:rank:empty}, \ref{item:rank:step}, and \ref{item:rank:submodularity2}, so they also define a matroid, although not all matroids are obtained this way.
\end{note}

\section{Closure}

Consider a set of vectors \(v_1, v_2, v_3, w\), and suppose you want to determine whether \(w \in \operatorname{span}(v_1, v_2, v_3)\).
In the language of matroids, this is equivalent to checking whether \(r(\{v_1, v_2, v_3, w\}) > r(\{v_1, v_2, v_3\})\).

\begin{definition}[Closure]
    For a subset \(X \subseteq E\), the \vocab{closure} of \(X\) in a matroid \(M\) is defined as
    \begin{equation}
        \operatorname{cl}_M(X) = \{e \in E : r(X + e) = r(X)\}.
    \end{equation}
\end{definition}

The closure can be thought of as analogous to the concept of span in vector spaces.

\begin{proposition} \label{prop:closure:inside}
    \(X \subseteq \operatorname{cl}_M(X)\) for all \(X \subseteq E\).
\end{proposition}

\begin{proof}
    Trivial.
\end{proof}

\begin{proposition} \label{prop:closure:monotonicity}
    If \(X \subseteq Y\), then \(\operatorname{cl}_M(X) \subseteq \operatorname{cl}_M(Y)\).
\end{proposition}

\begin{proof}
    Follows from \ref{lemma:diminishing}.
\end{proof}

\begin{proposition} \label{prop:closure:rank}
    If \(X \subseteq E\), then \(r(X) = r(\operatorname{cl}_M(X))\).
\end{proposition}

\begin{proof}
    Let \(Y\) be maximal such that \(X \subseteq Y \subseteq \operatorname{cl}_M(X)\) and \(r(Y) = r(X)\).
    If \(Y = \operatorname{cl}_M(X)\), we are done.
    Otherwise, let \(e \in \operatorname{cl}_M(X) - Y\).
    Since \(e \in \operatorname{cl}_M(X)\), we have \(r(X + e) = r(X)\).
    By the monotonicity of rank, \(r(Y + e) \geq r(Y)\), and since \(Y \subseteq \operatorname{cl}_M(X)\), we also have \(r(Y + e) \leq r(\operatorname{cl}_M(X)) = r(X)\).
    Therefore, \(r(Y + e) = r(Y)\), contradicting the maximality of \(Y\).
    Hence, \(Y = \operatorname{cl}_M(X)\), and \(r(\operatorname{cl}_M(X)) = r(X)\).
\end{proof}

\begin{proposition} \label{prop:closure:idempotent}
    Closure is idempotent,
    that is,
    \(\operatorname{cl}_M(\operatorname{cl}_M(X)) = \operatorname{cl}_M(X)\) for all \(X \subseteq E\).
\end{proposition}

\begin{proof}
    On the one hand, \(\operatorname{cl}_M(X) \subseteq \operatorname{cl}_M(\operatorname{cl}_M(X))\) by \ref{prop:closure:inside}.
    On the other hand, if \(e \in \operatorname{cl}_M(\operatorname{cl}_M(X))\),
    then \(r(X) = r(\operatorname{cl}_M(X)) = r(\operatorname{cl}_M(X) + e) \geq e(X + e)\),
    but also \(r(X) \leq r(X + e)\), so \(e \in \operatorname{cl}_M(X)\).
    Therefore, \(\operatorname{cl}_M(X) = \operatorname{cl}_M(\operatorname{cl}_M(X))\).
\end{proof}

\begin{proposition} \label{prop:closure:union}
    \(\operatorname{cl}_M( \operatorname{cl}_M(X) \cup \operatorname{cl}_M(Y) ) = \operatorname{cl}_M(X \cup Y)\) for all \(X, Y \subseteq E\).
\end{proposition}

\begin{proof}
    On the one hand, by monotonicity, we have \( \operatorname{cl}_M(X \cup Y) \subseteq \operatorname{cl}_M( \operatorname{cl}_M(X) \cup \operatorname{cl}_M(Y) )\).
    On the other hand, 
    \begin{equation}
        \operatorname{cl}_M( X \cup Y ) = \operatorname{cl}_M( \operatorname{cl}_M( X \cup Y ) ) \supseteq \operatorname{cl}_M( \operatorname{cl}_M(X) \cup \operatorname{cl}_M(Y) )
    \end{equation}
    because \(\operatorname{cl}_M(X) \subset \operatorname{cl}_M(X \cup Y)\) and \(\operatorname{cl}_M(Y) \subset \operatorname{cl}_M(X \cup Y)\).
\end{proof}

If you want to axiomatize closure, then you'd need Propositions~\ref{prop:closure:inside}, \ref{prop:closure:monotonicity}, \ref{prop:closure:idempotent}, and also a fourth axiom we won't prove here.

\section{Duality}

In this section, we define the dual of a matroid, explore its properties, and prove that duality is an involution.

\begin{definition}[Dual]
    Given a matroid \(M = (E, \mathcal{I})\),
    the \vocab{dual} of \(M\) is the matroid \(M^*\)
    whose bases are the complements of the bases of \(M\).
\end{definition}

The definition above specifies at most one matroid as the dual, but it is not immediately clear that the dual of a matroid is always a matroid. We will delay the proof of this fact until later. However, assuming the construction is valid, it follows directly that duality is an involution, i.e., \((M^*)^* = M\).

\subsection{Construction of the Dual Matroid}

Since a matroid is determined by its bases,
the above does uniquely define a matroid provided that it is actually a matroid.
We will prove it is a matroid by reconstructing its rank function.

Since \(r_M(X) = \max\{ |B \cap X| : B \text{ is a basis of } M\}\).
Therefore,
\begin{align*}
    r_{M^*}(X)
    &= \max\{ |B \cap X| : B \text{ is a basis of } M^*\} \\
    &= \max\{ |(E - B) \cap X| : B \text{ is a basis of } M\} \\
    &= \max\{ |X| - |B| + |B \cap (E - X)| : B \text{ is a basis of } M\} \\
    &= |X| - r(M) + \max\{ |B \cap (E - X)| : B \text{ is a basis of } M\} \\
    &= |X| - r(M) + r_M(E - X).
\end{align*}

It's really easy to prove that \(r'(X) = |X| - r_M(E) + r_M(E - X)\) satisfies \ref{item:rank:bounded}, \ref{item:rank:motonocity}, and \ref{item:rank:submodularity}.

\begin{proof}[Proof of \ref{item:rank:bounded}]
    \(r'(X) = |X| - r_M(E) + r_M(E - X) \leq |X|\) by \ref{item:rank:motonocity}.
\end{proof}

\begin{proof}[Proof of \ref{item:rank:motonocity}]
    If \(X \subseteq Y\),
    then
    \begin{align*}
        r'(Y) - r'(X)
        &= |Y - X| + r_M(E - Y) - r_M(E - X) \\
        &\geq r_M(Y - X) + r_M(E - Y) - r_M(E - X) \\
        &\geq r_M( (Y - X) \cup (E - Y) ) - r_M( E - X ) = 0.
    \end{align*}
\end{proof}

\begin{proof}[Proof of \ref{item:rank:submodularity}]
    Similar.
\end{proof}

\begin{proposition}
    \(B\) is a basis of \(M^*\) if and only if \(E - B\) is a basis of \(M\).
\end{proposition}

\begin{proof}
    \begin{align*}
        B \text{ is a basis of } M^*
        &\iff r_{M^*}(B) = r(M^*) \\
        &\iff |B| - r_M(E) + r_M(E - B) = |E| - r(M) \\
        &\iff r_M(E - B) = r(M) = |E - B| \\
        &\iff E - B \text{ is a basis of } M.
    \end{align*}
\end{proof}

The prefix \emph{co} dualizes a matroid term.
For example, \(I\) is \vocab{coindependent} if, and only if, \(|I| = r_{M^*}(I) = |I| - r(M) + r_M(E - I)\) if, and only if, \(r(M) = r_M(E - I)\).